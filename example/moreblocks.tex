% --------------------------------------------------------------------
% moreblocks.tex (Release 1.2)
% ==============------------------------------------------------------
% (C) in 2015/16 by Norman Markgraf
%  
% Release 1.0 nm (05.06.2015) Initial Concept
% Release 1.1 nm (05.06.2015) First real good Release
% Release 1.2 nm (24.01.2015) Added {.notblock} option.
% --------------------------------------------------------------------
\usepackage{etoolbox} % to use \undef{...}
\makeatletter
% --------------------------------------------------------------------
%
% Bemerkung(en)
%
\def\th@myremarkstyle{%
    \normalfont % body font
    \setbeamercolor{block title example}{bg=orange,fg=white}
    \setbeamercolor{block body example}{bg=orange!20,fg=black}
    \def\inserttheoremblockenv{exampleblock}
}
\theoremstyle{myremarkstyle}
\newtheorem*{Bemerkung}{Bemerkung}
\newtheorem*{Bemerkungen}{Bemerkungen}
% --------------------------------------------------------------------
%
% Beweis
%
\def\th@mybeweisstyle{%
    \normalfont % body font
    \setbeamercolor{block title example}{bg=blue,fg=white}
    \setbeamercolor{block body example}{bg=blue!20,fg=black}
    \def\inserttheoremblockenv{exampleblock}
}
\theoremstyle{mybeweisstyle}
\undef{\Beweis}
%\undef{\theorem}
\newtheorem{Beweis}{Beweis}

% --------------------------------------------------------------------
%
% Definition
%
\def\th@mydefinitionstyle{%
    \normalfont % body font
    \setbeamercolor{block title example}{bg=red,fg=white}
    \setbeamercolor{block body example}{bg=red!20,fg=black}
    \def\inserttheoremblockenv{exampleblock}
}
\theoremstyle{mydefinitionstyle}
\undef{\Definition}
\undef{\definition}
\newtheorem{Definition}{Definition}
\newtheorem{definition}{Definition}
% --------------------------------------------------------------------
%
% Satz
%
\def\th@mysatzstyle{%
    \normalfont % body font
    \setbeamercolor{block title example}{bg=blue,fg=white}
    \setbeamercolor{block body example}{bg=blue!20,fg=black}
    \def\inserttheoremblockenv{exampleblock}
}
\theoremstyle{mysatzstyle}
\undef{\Satz}
%\undef{\theorem}
\newtheorem{Satz}{Satz}
%\newtheorem{definition}{Definition}
% --------------------------------------------------------------------
%
% Lemma
%
\def\th@mylemmastyle{%
    \normalfont % body font
    \setbeamercolor{block title example}{bg=blue!60,fg=darkgray!20}
    \setbeamercolor{block body example}{bg=blue!5,fg=black}
    \def\inserttheoremblockenv{exampleblock}
}
\theoremstyle{mylemmastyle}
\undef{\Lemma}
\newtheorem{Lemma}{Lemma}
% --------------------------------------------------------------------
%
% Lösung
%
\def\th@mysolutionstyle{%
    \normalfont % body font
    \setbeamercolor{block title example}{bg=green!60,fg=darkgray!5}
    \setbeamercolor{block body example}{bg=green!5,fg=black}
    \def\inserttheoremblockenv{exampleblock}
}
\theoremstyle{mysolutionstyle}
\undef{\Loesung}
\newtheorem{Loesung}{Lösung}
% --------------------------------------------------------------------
%
% Übung
%
\def\th@myexercisestyle{%
    \normalfont % body font
    \setbeamercolor{block title example}{bg=green!75,fg=darkgray!5}
    \setbeamercolor{block body example}{bg=green!15,fg=black}
    \def\inserttheoremblockenv{exampleblock}
}
\theoremstyle{myexercisestyle}
\undef{\Uebung}
\newtheorem{Uebung}{Übung}
% --------------------------------------------------------------------
%
% Fakt
%
\def\th@myfactstyle{%
    \normalfont % body font
    \setbeamercolor{block title example}{bg=green!60,fg=darkgray!5}
    \setbeamercolor{block body example}{bg=green!5,fg=black}
    \def\inserttheoremblockenv{exampleblock}
}
\theoremstyle{myfactstyle}
\undef{\Fakt}
\newtheorem{Fakt}{Fakt}
\undef{\Fakten}
\newtheorem{Fakten}{Fakten}
% --------------------------------------------------------------------
\makeatother
