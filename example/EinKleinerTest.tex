\PassOptionsToPackage{unicode=true}{hyperref} % options for packages loaded elsewhere
\PassOptionsToPackage{hyphens}{url}
%
\documentclass[ignorenonframetext,]{beamer}
\setbeamertemplate{caption}[numbered]
\setbeamertemplate{caption label separator}{: }
\setbeamercolor{caption name}{fg=normal text.fg}
\beamertemplatenavigationsymbolsempty
\usepackage{lmodern}
\usepackage{amssymb,amsmath}
\usepackage{ifxetex,ifluatex}
\usepackage{fixltx2e} % provides \textsubscript
\ifnum 0\ifxetex 1\fi\ifluatex 1\fi=0 % if pdftex
  \usepackage[T1]{fontenc}
  \usepackage[utf8]{inputenc}
  \usepackage{textcomp} % provides euro and other symbols
\else % if luatex or xelatex
  \usepackage{unicode-math}
  \defaultfontfeatures{Ligatures=TeX,Scale=MatchLowercase}
\fi
% use upquote if available, for straight quotes in verbatim environments
\IfFileExists{upquote.sty}{\usepackage{upquote}}{}
% use microtype if available
\IfFileExists{microtype.sty}{%
\usepackage[]{microtype}
\UseMicrotypeSet[protrusion]{basicmath} % disable protrusion for tt fonts
}{}
\IfFileExists{parskip.sty}{%
\usepackage{parskip}
}{% else
\setlength{\parindent}{0pt}
\setlength{\parskip}{6pt plus 2pt minus 1pt}
}
\usepackage{hyperref}
\hypersetup{
            pdftitle={Ein kleiner Test},
            pdfauthor={Norman Markgraf},
            pdfborder={0 0 0},
            breaklinks=true}
\urlstyle{same}  % don't use monospace font for urls
\newif\ifbibliography
% Prevent slide breaks in the middle of a paragraph:
\widowpenalties 1 10000
\raggedbottom
\setbeamertemplate{part page}{
\centering
\begin{beamercolorbox}[sep=16pt,center]{part title}
  \usebeamerfont{part title}\insertpart\par
\end{beamercolorbox}
}
\setbeamertemplate{section page}{
\centering
\begin{beamercolorbox}[sep=12pt,center]{part title}
  \usebeamerfont{section title}\insertsection\par
\end{beamercolorbox}
}
\setbeamertemplate{subsection page}{
\centering
\begin{beamercolorbox}[sep=8pt,center]{part title}
  \usebeamerfont{subsection title}\insertsubsection\par
\end{beamercolorbox}
}
\AtBeginPart{
  \frame{\partpage}
}
\AtBeginSection{
  \ifbibliography
  \else
    \frame{\sectionpage}
  \fi
}
\AtBeginSubsection{
  \frame{\subsectionpage}
}
\setlength{\emergencystretch}{3em}  % prevent overfull lines
\providecommand{\tightlist}{%
  \setlength{\itemsep}{0pt}\setlength{\parskip}{0pt}}
\setcounter{secnumdepth}{0}

% set default figure placement to htbp
\makeatletter
\def\fps@figure{htbp}
\makeatother

\usepackage{xspace}
\usepackage{tikz}
\usetikzlibrary{positioning}

\newcommand{\Sinnspruch}{\relax}

%\AtBeginSection[]%
%{%
%\begin{frame}[plain]%
%
%  \tikz[remember picture] \node[circle] (a) {};
%  
%  \begin{center}%
%    \usebeamerfont{section title}\insertsection%
%  \end{center}%
%\end{frame}%
%}

\title{Ein kleiner Test}
\author{Norman Markgraf}
\date{18 Dezember 2017}

\begin{document}
\frame{\titlepage}

\hypertarget{test}{%
\section{Test}\label{test}}

\begin{frame}{%
\protect\hypertarget{ein-paar-testsenarien}{%
Ein paar Testsenarien}}


\begin{center}

Das ist mittig!

\end{center}

Das ist normaler Text!

{\Large{}

Das ist GROSS!

}

Das ist normaler Text!


\begin{center}
{\Large{}

Das ist GROSS!

}\end{center}

Das ist normaler Text!


\begin{center}
{\LARGE{}

Das ist GROSS!

}\end{center}

Das ist normaler Text!

\end{frame}

\begin{frame}{%
\protect\hypertarget{nun-spans-statt-divs}{%
Nun Spans statt Divs}}

Das ist {{\small{}ein kleiner}} {{\Large{}GROSSER}} Test!

\end{frame}

\begin{frame}{%
\protect\hypertarget{alle-auf-einer-seite}{%
Alle auf einer Seite:}}

{{\tiny{}tiny}} {{\scriptsize{}scriptsize}}
{{\footnotesize{}footnotesize}} {{\small{}small}} (default)
{{\normalsize{}normalsize}} {{\Large{}large}} {{\Large{}Large}}
{{\huge{}huge}} {{\Huge{}Huge}}

\end{frame}

\hypertarget{nun-einmal-ein-sinnspruch-im-section-title}{%
\section{Nun einmal ein Sinnspruch im
Section-Title!}\label{nun-einmal-ein-sinnspruch-im-section-title}}


\mode<all>\begin{quote}\small 

Das hier ist ein Sinnspruch und sollte als solcher.

Auch genau so behandelt werden.

{{\scriptsize – Norman Markgraf}}

\end{quote}
\mode<*>

\begin{frame}{%
\protect\hypertarget{ein-sinnspruch-im-normalen-frame}{%
Ein Sinnspruch im normalen Frame}}


\mode<all>\begin{quote}\small 

Das hier ist ein Sinnspruch und sollte als solcher.

Auch genau so behandelt werden.

{{\scriptsize – Norman Markgraf}}

\end{quote}
\mode<*>

\end{frame}

\end{document}
