\documentclass[]{article}
\usepackage{lmodern}
\usepackage{amssymb,amsmath}
\usepackage{ifxetex,ifluatex}
\usepackage{fixltx2e} % provides \textsubscript
\ifnum 0\ifxetex 1\fi\ifluatex 1\fi=0 % if pdftex
  \usepackage[T1]{fontenc}
  \usepackage[utf8]{inputenc}
\else % if luatex or xelatex
  \ifxetex
    \usepackage{mathspec}
  \else
    \usepackage{fontspec}
  \fi
  \defaultfontfeatures{Ligatures=TeX,Scale=MatchLowercase}
\fi
% use upquote if available, for straight quotes in verbatim environments
\IfFileExists{upquote.sty}{\usepackage{upquote}}{}
% use microtype if available
\IfFileExists{microtype.sty}{%
\usepackage{microtype}
\UseMicrotypeSet[protrusion]{basicmath} % disable protrusion for tt fonts
}{}
\usepackage[margin=1in]{geometry}
\usepackage{hyperref}
\hypersetup{unicode=true,
            pdftitle={Ein kleiner Test},
            pdfauthor={Norman Markgraf},
            pdfborder={0 0 0},
            breaklinks=true}
\urlstyle{same}  % don't use monospace font for urls
\usepackage{graphicx,grffile}
\makeatletter
\def\maxwidth{\ifdim\Gin@nat@width>\linewidth\linewidth\else\Gin@nat@width\fi}
\def\maxheight{\ifdim\Gin@nat@height>\textheight\textheight\else\Gin@nat@height\fi}
\makeatother
% Scale images if necessary, so that they will not overflow the page
% margins by default, and it is still possible to overwrite the defaults
% using explicit options in \includegraphics[width, height, ...]{}
\setkeys{Gin}{width=\maxwidth,height=\maxheight,keepaspectratio}
\IfFileExists{parskip.sty}{%
\usepackage{parskip}
}{% else
\setlength{\parindent}{0pt}
\setlength{\parskip}{6pt plus 2pt minus 1pt}
}
\setlength{\emergencystretch}{3em}  % prevent overfull lines
\providecommand{\tightlist}{%
  \setlength{\itemsep}{0pt}\setlength{\parskip}{0pt}}
\setcounter{secnumdepth}{0}
% Redefines (sub)paragraphs to behave more like sections
\ifx\paragraph\undefined\else
\let\oldparagraph\paragraph
\renewcommand{\paragraph}[1]{\oldparagraph{#1}\mbox{}}
\fi
\ifx\subparagraph\undefined\else
\let\oldsubparagraph\subparagraph
\renewcommand{\subparagraph}[1]{\oldsubparagraph{#1}\mbox{}}
\fi

%%% Use protect on footnotes to avoid problems with footnotes in titles
\let\rmarkdownfootnote\footnote%
\def\footnote{\protect\rmarkdownfootnote}

%%% Change title format to be more compact
\usepackage{titling}

% Create subtitle command for use in maketitle
\providecommand{\subtitle}[1]{
  \posttitle{
    \begin{center}\large#1\end{center}
    }
}

\setlength{\droptitle}{-2em}

  \title{Ein kleiner Test}
    \pretitle{\vspace{\droptitle}\centering\huge}
  \posttitle{\par}
    \author{Norman Markgraf}
    \preauthor{\centering\large\emph}
  \postauthor{\par}
      \predate{\centering\large\emph}
  \postdate{\par}
    \date{28 Oktober 2018}

\usepackage{xspace}
\include{heaader.tex}

\begin{document}
\maketitle

\hypertarget{test}{%
\section{Test}\label{test}}

\hypertarget{ein-paar-testszenarien-div-blocke}{%
\subsection{Ein paar Testszenarien
(DIV-Blöcke)}\label{ein-paar-testszenarien-div-blocke}}


\begin{center}

Das ist mittig!


\end{center}

Das ist normaler Text!

{\Large{}

Das ist GROSS!

}

Das ist normaler Text!


\begin{center}
{\Large{}

Das ist GROSS!

}
\end{center}

Das ist normaler Text!


\begin{center}
{\LARGE{}

Das ist GROSS!

}
\end{center}

Das ist normaler Text!

\hypertarget{nun-spans-statt-divs}{%
\subsection{Nun Spans statt Divs}\label{nun-spans-statt-divs}}

Das ist {{\small{}ein kleiner}} {{\Large{}GROSSER}} Test!

\hypertarget{alle-auf-einer-seite}{%
\subsection{Alle auf einer Seite:}\label{alle-auf-einer-seite}}

{{\tiny{}tiny}} {{\scriptsize{}scriptsize}}
{{\footnotesize{}footnotesize}} {{\small{}small}} (default)
{{\normalsize{}normalsize}} {{\Large{}large}} {{\Large{}Large}}
{{\huge{}huge}} {{\Huge{}Huge}}

{{\normalfont{}normal}} {{\rmfamily{}roman}} {{\sffamily{}sanserif}}
{{\ttfamily{}teletype}} \textsc{smallcaps} {{\slshape{}slanted}}
{{\upshape{}upright}} {{\itshape{}italic}}

\hypertarget{justified-alignments}{%
\subsection{Justified Alignments}\label{justified-alignments}}

All small:

{\small{}

Auch gibt es niemanden, der den Schmerz an sich liebt, sucht oder
wünscht, nur, weil er Schmerz ist, es sei denn, es kommt zu zufälligen
Umständen, in denen Mühen und Schmerz ihm große Freude bereiten können.

Justified Left:


\begin{flushright}

Um ein triviales Beispiel zu nehmen, wer von uns unterzieht sich je
anstrengender körperlicher Betätigung, außer um Vorteile daraus zu
ziehen? Aber wer hat irgend ein Recht, einen Menschen zu tadeln, der die
Entscheidung trifft, eine Freude zu genießen, die keine unangenehmen
Folgen hat, oder einen, der Schmerz vermeidet, welcher keine daraus
resultierende Freude nach sich zieht?


\end{flushright}

Justified Right:


\begin{flushleft}

Auch gibt es niemanden, der den Schmerz an sich liebt, sucht oder
wünscht, nur, weil er Schmerz ist, es sei denn, es kommt zu zufälligen
Umständen, in denen Mühen und Schmerz ihm große Freude bereiten können.
Um ein triviales Beispiel zu nehmen, wer von uns unterzieht sich je
anstrengender körperlicher Betätigung, außer um Vorteile daraus zu
ziehen?


\end{flushleft}

Normal, but small:

Aber wer hat irgend ein Recht, einen Menschen zu tadeln, der die
Entscheidung trifft, eine Freude zu genießen, die keine unangenehmen
Folgen hat, oder einen, der Schmerz vermeidet, welcher keine daraus
resultierende Freude nach sich zieht?Auch gibt es niemanden, der den
Schmerz an sich liebt, sucht oder wünscht, nur,

}

\hypertarget{lucken-fur-losungen}{%
\subsection{Lücken für Lösungen}\label{lucken-fur-losungen}}

Berechnen Sie die folgenden Aufgaben:

\begin{itemize}
\tightlist
\item
  \(1+2+3+4=\;\){\(10\)}
\item
  \(2+3+4+5=\;\){\(14\)}
\end{itemize}


\end{document}
