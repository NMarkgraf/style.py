% Options for packages loaded elsewhere
\PassOptionsToPackage{unicode}{hyperref}
\PassOptionsToPackage{hyphens}{url}
%
\documentclass[
  ignorenonframetext,
]{beamer}
\usepackage{pgfpages}
\setbeamertemplate{caption}[numbered]
\setbeamertemplate{caption label separator}{: }
\setbeamercolor{caption name}{fg=normal text.fg}
\beamertemplatenavigationsymbolsempty
% Prevent slide breaks in the middle of a paragraph
\widowpenalties 1 10000
\raggedbottom
\setbeamertemplate{part page}{
  \centering
  \begin{beamercolorbox}[sep=16pt,center]{part title}
    \usebeamerfont{part title}\insertpart\par
  \end{beamercolorbox}
}
\setbeamertemplate{section page}{
  \centering
  \begin{beamercolorbox}[sep=12pt,center]{part title}
    \usebeamerfont{section title}\insertsection\par
  \end{beamercolorbox}
}
\setbeamertemplate{subsection page}{
  \centering
  \begin{beamercolorbox}[sep=8pt,center]{part title}
    \usebeamerfont{subsection title}\insertsubsection\par
  \end{beamercolorbox}
}
\AtBeginPart{
  \frame{\partpage}
}
\AtBeginSection{
  \ifbibliography
  \else
    \frame{\sectionpage}
  \fi
}
\AtBeginSubsection{
  \frame{\subsectionpage}
}
\usepackage{lmodern}
\usepackage{amssymb,amsmath}
\usepackage{ifxetex,ifluatex}
\ifnum 0\ifxetex 1\fi\ifluatex 1\fi=0 % if pdftex
  \usepackage[T1]{fontenc}
  \usepackage[utf8]{inputenc}
  \usepackage{textcomp} % provide euro and other symbols
\else % if luatex or xetex
  \usepackage{unicode-math}
  \defaultfontfeatures{Scale=MatchLowercase}
  \defaultfontfeatures[\rmfamily]{Ligatures=TeX,Scale=1}
\fi
% Use upquote if available, for straight quotes in verbatim environments
\IfFileExists{upquote.sty}{\usepackage{upquote}}{}
\IfFileExists{microtype.sty}{% use microtype if available
  \usepackage[]{microtype}
  \UseMicrotypeSet[protrusion]{basicmath} % disable protrusion for tt fonts
}{}
\makeatletter
\@ifundefined{KOMAClassName}{% if non-KOMA class
  \IfFileExists{parskip.sty}{%
    \usepackage{parskip}
  }{% else
    \setlength{\parindent}{0pt}
    \setlength{\parskip}{6pt plus 2pt minus 1pt}}
}{% if KOMA class
  \KOMAoptions{parskip=half}}
\makeatother
\usepackage{xcolor}
\IfFileExists{xurl.sty}{\usepackage{xurl}}{} % add URL line breaks if available
\IfFileExists{bookmark.sty}{\usepackage{bookmark}}{\usepackage{hyperref}}
\hypersetup{
  pdftitle={Ein kleiner Test},
  pdfauthor={Norman Markgraf},
  hidelinks,
  pdfcreator={LaTeX via pandoc}}
\urlstyle{same} % disable monospaced font for URLs
\newif\ifbibliography
\setlength{\emergencystretch}{3em} % prevent overfull lines
\providecommand{\tightlist}{%
  \setlength{\itemsep}{0pt}\setlength{\parskip}{0pt}}
\setcounter{secnumdepth}{-\maxdimen} % remove section numbering
\usepackage{xcolor}
% --------------------------------------------------------------------
% moreblocks.tex (Release 1.2)
% ==============------------------------------------------------------
% (C) in 2015/16 by Norman Markgraf
%  
% Release 1.0 nm (05.06.2015) Initial Concept
% Release 1.1 nm (05.06.2015) First real good Release
% Release 1.2 nm (24.01.2015) Added {.notblock} option.
% --------------------------------------------------------------------
\usepackage{etoolbox} % to use \undef{...}
\makeatletter
% --------------------------------------------------------------------
%
% Bemerkung(en)
%
\def\th@myremarkstyle{%
    \normalfont % body font
    \setbeamercolor{block title example}{bg=orange,fg=white}
    \setbeamercolor{block body example}{bg=orange!20,fg=black}
    \def\inserttheoremblockenv{exampleblock}
}
\theoremstyle{myremarkstyle}
\newtheorem*{Bemerkung}{Bemerkung}
\newtheorem*{Bemerkungen}{Bemerkungen}
% --------------------------------------------------------------------
%
% Beweis
%
\def\th@mybeweisstyle{%
    \normalfont % body font
    \setbeamercolor{block title example}{bg=blue,fg=white}
    \setbeamercolor{block body example}{bg=blue!20,fg=black}
    \def\inserttheoremblockenv{exampleblock}
}
\theoremstyle{mybeweisstyle}
\undef{\Beweis}
%\undef{\theorem}
\newtheorem{Beweis}{Beweis}

% --------------------------------------------------------------------
%
% Definition
%
\def\th@mydefinitionstyle{%
    \normalfont % body font
    \setbeamercolor{block title example}{bg=red,fg=white}
    \setbeamercolor{block body example}{bg=red!20,fg=black}
    \def\inserttheoremblockenv{exampleblock}
}
\theoremstyle{mydefinitionstyle}
\undef{\Definition}
\undef{\definition}
\newtheorem{Definition}{Definition}
\newtheorem{definition}{Definition}
% --------------------------------------------------------------------
%
% Satz
%
\def\th@mysatzstyle{%
    \normalfont % body font
    \setbeamercolor{block title example}{bg=blue,fg=white}
    \setbeamercolor{block body example}{bg=blue!20,fg=black}
    \def\inserttheoremblockenv{exampleblock}
}
\theoremstyle{mysatzstyle}
\undef{\Satz}
%\undef{\theorem}
\newtheorem{Satz}{Satz}
%\newtheorem{definition}{Definition}
% --------------------------------------------------------------------
%
% Lemma
%
\def\th@mylemmastyle{%
    \normalfont % body font
    \setbeamercolor{block title example}{bg=blue!60,fg=darkgray!20}
    \setbeamercolor{block body example}{bg=blue!5,fg=black}
    \def\inserttheoremblockenv{exampleblock}
}
\theoremstyle{mylemmastyle}
\undef{\Lemma}
\newtheorem{Lemma}{Lemma}
% --------------------------------------------------------------------
%
% Lösung
%
\def\th@mysolutionstyle{%
    \normalfont % body font
    \setbeamercolor{block title example}{bg=green!60,fg=darkgray!5}
    \setbeamercolor{block body example}{bg=green!5,fg=black}
    \def\inserttheoremblockenv{exampleblock}
}
\theoremstyle{mysolutionstyle}
\undef{\Loesung}
\newtheorem{Loesung}{Lösung}
% --------------------------------------------------------------------
%
% Übung
%
\def\th@myexercisestyle{%
    \normalfont % body font
    \setbeamercolor{block title example}{bg=green!75,fg=darkgray!5}
    \setbeamercolor{block body example}{bg=green!15,fg=black}
    \def\inserttheoremblockenv{exampleblock}
}
\theoremstyle{myexercisestyle}
\undef{\Uebung}
\newtheorem{Uebung}{Übung}
% --------------------------------------------------------------------
%
% Fakt
%
\def\th@myfactstyle{%
    \normalfont % body font
    \setbeamercolor{block title example}{bg=green!60,fg=darkgray!5}
    \setbeamercolor{block body example}{bg=green!5,fg=black}
    \def\inserttheoremblockenv{exampleblock}
}
\theoremstyle{myfactstyle}
\undef{\Fakt}
\newtheorem{Fakt}{Fakt}
\undef{\Fakten}
\newtheorem{Fakten}{Fakten}
% --------------------------------------------------------------------
\makeatother

\newcommand{\cemph}{\color{green}}
\newcommand{\cstrong}{\color{red}}
\usepackage{xspace}
\usepackage{xspace}


\title{Ein kleiner Test}
\subtitle{Beamer - Fassung}
\author{Norman Markgraf}
\date{08 Juli 2019}

\begin{document}
\frame{\titlepage}

\hypertarget{test}{%
\section{Test}\label{test}}

\begin{frame}{Ein paar Testszenarien}
\protect\hypertarget{ein-paar-testszenarien}{}


\begin{center}

Das ist mittig!


\end{center}

Das ist normaler Text!

{\Large{}

Das ist GROSS!

}

Das ist normaler Text!


\begin{center}
{\Large{}

Das ist GROSS!

}
\end{center}

Das ist normaler Text!


\begin{center}
{\LARGE{}

Das ist GROSS!

}
\end{center}

Das ist normaler Text!

\end{frame}

\begin{frame}{Nun Spans statt Divs}
\protect\hypertarget{nun-spans-statt-divs}{}

Das ist {{\tiny{}ein sehr kleiner}}, {{\small{}kleiner}} und
{{\Large{}ein GROSSER}} Test!

Manchmal möchte man {{\cemph{}{\scriptsize{}klein und grün}}} und
{{\cstrong{}{\huge{}groß und rot}}} schreiben.

\end{frame}

\begin{frame}{Alle auf einer Seite:}
\protect\hypertarget{alle-auf-einer-seite}{}

{{\tiny{}tiny}} {{\scriptsize{}scriptsize}}
{{\footnotesize{}footnotesize}} {{\small{}small}} (default)
{{\normalsize{}normalsize}} {{\Large{}large}} {{\Large{}Large}}
{{\huge{}huge}} {{\Huge{}Huge}}

{{\normalfont{}normal}} {{\rmfamily{}roman}} {{\sffamily{}sanserif}}
{{\ttfamily{}teletype}} \textsc{smallcaps} {{\slshape{}slanted}}
{{\upshape{}upright}} {{\itshape{}italic}}

\end{frame}

\begin{frame}{Justified Alignments}
\protect\hypertarget{justified-alignments}{}

{\tiny{}

Auch gibt es niemanden, der den Schmerz an sich liebt, sucht oder
wünscht, nur, weil er Schmerz ist, es sei denn, es kommt zu zufälligen
Umständen, in denen Mühen und Schmerz ihm große Freude bereiten können.


\begin{flushright}

Um ein triviales Beispiel zu nehmen, wer von uns unterzieht sich je
anstrengender körperlicher Betätigung, außer um Vorteile daraus zu
ziehen? Aber wer hat irgend ein Recht, einen Menschen zu tadeln, der die
Entscheidung trifft, eine Freude zu genießen, die keine unangenehmen
Folgen hat, oder einen, der Schmerz vermeidet, welcher keine daraus
resultierende Freude nach sich zieht?


\end{flushright}


\begin{flushleft}

Auch gibt es niemanden, der den Schmerz an sich liebt, sucht oder
wünscht, nur, weil er Schmerz ist, es sei denn, es kommt zu zufälligen
Umständen, in denen Mühen und Schmerz ihm große Freude bereiten können.
Um ein triviales Beispiel zu nehmen, wer von uns unterzieht sich je
anstrengender körperlicher Betätigung, außer um Vorteile daraus zu
ziehen?


\end{flushleft}

Aber wer hat irgend ein Recht, einen Menschen zu tadeln, der die
Entscheidung trifft, eine Freude zu genießen, die keine unangenehmen
Folgen hat, oder einen, der Schmerz vermeidet, welcher keine daraus
resultierende Freude nach sich zieht?Auch gibt es niemanden, der den
Schmerz an sich liebt, sucht oder wünscht, nur,

}

\end{frame}

\begin{frame}{Lücken für Lösungen}
\protect\hypertarget{lucken-fur-losungen}{}

Berechnen Sie die folgenden Aufgaben:

\begin{itemize}
\tightlist
\item
  \(1+2+3+4=\;\){\(10\)}
\item
  \(2+3+4+5=\;\){\(14\)}
\end{itemize}

\end{frame}

\hypertarget{nun-einmal-ein-sinnspruch-im-section-title}{%
\section{Nun einmal ein Sinnspruch im
Section-Title!}\label{nun-einmal-ein-sinnspruch-im-section-title}}

\begin{frame}{Nun einmal ein Sinnspruch im Section-Title!}


\begin{quote}\small{}

Das hier ist ein Sinnspruch und sollte als solcher.

Auch genau so behandelt werden.

{
{\scriptsize{} --\xspace{} -- Norman Markgraf}
}

\end{quote}

\end{frame}

\begin{frame}{Ein Sinnspruch im normalen Frame}
\protect\hypertarget{ein-sinnspruch-im-normalen-frame}{}


\begin{quote}\small{}

Das hier ist ein Sinnspruch und sollte als solcher.

Auch genau so behandelt werden.

{
{\scriptsize{} --\xspace{} -- Norman Markgraf}
}

\end{quote}

\end{frame}

\begin{frame}{Ein paar der alten ``moreblock'' Sachen (I/II)}
\protect\hypertarget{ein-paar-der-alten-moreblock-sachen-iii}{}


\begin{Satz}[von Düsterloh]

Das Niveau hat keine untere Schranke.

\end{Satz}


\begin{Beweis}[des Satzes von Düsterloh]

Donald Trump.

\end{Beweis}


\begin{Bemerkung}[Notwendig dafür ist]

Das wir ``moreblock.tex'' einbinden!

\end{Bemerkung}


\begin{Beispiel}

Das Beispiel sehen wir hier!

\end{Beispiel}

\end{frame}

\begin{frame}{Ein paar der alten ``moreblock'' Sachen (II/II)}
\protect\hypertarget{ein-paar-der-alten-moreblock-sachen-iiii}{}


\begin{definition}

Es ist, also muss es!

\end{definition}


\begin{Uebung}

Eine Übung zur rechten Zeit, und wir wissen was übrig bleibt.

\end{Uebung}


\begin{Fakt}

Das ist so. Das bleibt so.

\end{Fakt}

\end{frame}

\begin{frame}{Ende!}
\protect\hypertarget{ende}{}

\end{frame}

\end{document}
